\section{Einleitung}
Diese Bachelorarbeit ist aus einer Projektarbeit entstanden, welche im Semester 2014/15 stattgefunden hat.
In dieser Projektarbeit habe ich einen Algorithmus aus dem Paper "'Capture of an Intruder by Mobile Agents"' \cite{cima_paper} in einem Java-Applet visuell dargestellt und implementiert. Es handelt sich bei dem Algorithmus um eine Variante des sogenannten "'graph-searching problems"'.
\\
\\
Man kann sich das Problem anschaulich wie folgt vorstellen: Ein Einbrecher ist in ein Netzwerk eingedrungen und soll nun von mobilen Agenten gesucht werden. Sowohl die Agenten, als auch der Einbrecher können sich nur auf den Kanten des Graphen bewegen. Der Algorithmus soll die minimale Anzahl der Agenten berechnen, die nötig sind, um das gesamte Netzwerk systematisch nach dem Einbrecher zu durchsuchen. Ein Knoten, auf dem Agenten sind bzw. waren ist dekontaminiert. Ein Knoten, in dem noch kein Agent gewesen ist, wird als kontaminiert bezeichnet. Außerdem gibt der Algorithmus den Startpunkt im Netzwerk an, von dem die Agenten die Suche anfangen sollen (die sogenannte Homebase, siehe auch Definition \ref{def_homebase}). Je nachdem welcher Punkt des Netzwerks als Homebase bestimmt wird, kann sich die Anzahl der benötigten Agenten ändern. Es wird daher die Homebase mit der minimalen Agentenzahl gesucht.
\\
\\
In der im Paper beschriebenen Variante des "'graph-searching problems"' hat jede Kante ein Gewicht $\geq 1$. Dieses Gewicht sagt aus, wie viele Agenten mindestens über eine Kante laufen müssen, um diese zu dekontaminieren. Anschaulich kann man sich die Kante als Gang in einem Haus vorstellen. Je nachdem, wie groß und verwinkelt dieser Gang ist, braucht man verschieden viele Agenten, um diesen Gang zu dekontaminieren.
\\
\\
Da dieses Problem NP-vollständig auf allgemeinen Graphen ist, werden sowohl im Paper, als auch in dieser Bachelorarbeit, nur Bäume betrachtet. Dadurch ist es möglich, einen linearen Algorithmus anzugeben, in dem die minimale Anzahl an Agenten sowie die dazugehörige Homebase berechnet werden können.
\\
\\
Eine wichtige Eigenschaft, die wir aufrecht erhalten wollen, ist die Monotonie. 
\begin{mydef}\label{def_monotonie}
	Monotonie bedeutet, dass ein bereits dekontaminierter Knoten nicht mehr kontaminiert werden darf. Der Baum wird also sukzessive dekontaminiert. 
\end{mydef}
 Um diese Eigenschaft zu gewährleisten, müssen wir sogenannte Wachen aufstellen, die bereits dekontaminierte Teilbäume vor dem Einbrecher schützen. Die Anzahl der benötigten Wachen hängt davon ab, wie groß das Kantengewicht der angrenzenden Teilbäume ist. Gibt es keinen benachbarten Teilbaum, so wird an dieser Stelle keine Wache benötigt.\\
Die Agenten dekontaminieren die einzelne Teilbäume sukzessive, bis der gesamte Baum bearbeitet wurde und somit dekontaminiert ist. Der Algorithmus ist danach abgeschlossen.
\\
\\
Das Ziel dieser Bachelorarbeit ist es, auf Grundlage des in "'Capture of an Intruder by Mobile Agents"' \cite{cima_paper} beschriebenen Algorithmus, weitere Problemstellungen zu betrachten. Ich untersuche im folgenden, wie sich ein sogenanntes Potenzial auf den Algorithmus auswirkt. Das Potenzial gibt an, um wie viel eine oder mehrere Kanten(-gewichte) reduziert werden können. Ich entwickle nun einen Algorithmus für verschiedene Varianten des Potenzial-Problems. Der Algorithmus kann dabei ein Potenzial nutzt um einzelne Kantengewichte zu verringern. Dadurch sollen die für die Dekontaminierung benötigte Anzahl an Agenten verringert werden. Allerdings ändert sich die minimale Anzahl der Agenten unterschiedliche stark, je nachdem, welche Kanten durch das Potenzial reduziert werden, weshalb das Potential optimal benutzt werden soll, um die Agentenzahl möglichst stark zu senken.
\\

