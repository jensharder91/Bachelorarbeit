\section{Fazit}


Für vereinfachte Problemstellungen ($k = 1$ und $k \geq 1$ auf einer Kante) im Zusammenhang mit der Anwendung eines Potentials auf Bäumen wurden in dieser Bachelorarbeit Erweiterungen des zu Grunde liegenden Algorithmus entwickelt, um in linearer Zeit eine der Kanten zu berechnen, auf der das Potential angewendet werden sollte. Außerdem konnten diese Algorithmen in einem Applet implementiert werden, um so eine visuelle Darstellung der Algorithmen zu ermöglichen.
\\
\\
In dieser Arbeit konnte begründet werden, weshalb es nötig sein kann, das Potential auf bis zu linear viele Kanten zu verteilen. Allerdings stellte es sich als schwierig heraus, die Kanten zu berechnen, auf die das Potential verteilt werden sollte und es war nicht möglich im Rahmen dieser Bachelorarbeit zu dieser Problemstellung einen Algorithmus anzugeben. Diese Variante des Potentialproblems könnte in zukünftigen Arbeiten weiter untersucht werden, um einen effizienten Algorithmus zu finden, falls dies möglich ist.
