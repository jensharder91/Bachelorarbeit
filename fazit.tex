\section{Fazit}


Für vereinfachte Problemstellungen ($k = 1$ und $k \geq 1$ auf einer Kante) im Zusammenhang mit einer Anwendung eines Potentials auf Bäumen konnten in dieser Bachelorarbeit Erweiterungen des zu Grunde liegenden Algorithmus entwickelt werden, um in linearer Zeit eine der Kanten zu berechnen, auf der das Potential angewendet werden sollte. Außerdem konnten diese Algorithmen in einem Applet implementiert werden, um so eine visuelle Darstellung der Algorithmen zu ermöglichen.
\\
\\
In dieser Arbeit konnte begründet werden, weshalb es nötig sein kann, das Potential auf bis zu linear viele Kanten zu verteilen. Allerdings stellte es sich als schwierig heraus, die Kanten zu berechnen, auf die das Potential verteilt werden sollte und es konnte im Rahmen dieser Bachelorarbeit zu dieser Problemstellung kein Algorithmus angegeben werden. 