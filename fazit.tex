\section{Fazit}

Als Grundlage dieser Bachelorarbeit wurde der Algorithmus aus \cite{cima_paper} in Kapitel \ref{paperAlgoChapter} beschrieben und erläutert. Aufbauend auf diesem Algorithmus konnten einige Erweiterungen entwickelt bzw. neue Aspekte betrachtet werden. Diese sind
\begin{itemize}
	\item Modifikation des Algorithmus in Kapitel \ref{modifizierterAlgoChapter}
	\item Potentialproblem $k = 1$ in Kapitel \ref{kap_pot=1}
	\item Potentialproblem $k \geq 1$ auf einer Kante in Kapitel \ref{kap_pot>=1}
	\item Potentialproblem $k > 1$ auf verteilten Kanten in Kapitel \ref{kap_pot>1_verteilt}
\end{itemize}

Für zwei der Problemstellungen ($k = 1$ und $k \geq 1$ auf einer Kante)  konnte der modifizierte Algorithmus im Zusammenhang mit der Anwendung eines Potentials auf Bäumen erweitert werden, um in linearer Zeit eine der Kanten zu berechnen, auf der das Potential angewendet werden sollte. Außerdem konnten diese beiden Algorithmen mit der entsprechenden Erweiterung in einem Applet implementiert werden, um so eine visuelle Darstellung der Algorithmen zu ermöglichen (siehe Kapitel \ref{kap_implementierung}). 
\\
\\
In dieser Arbeit konnte außerdem gezeigt werden, weshalb es nötig sein kann, das Potential auf bis zu linear viele Kanten zu verteilen ($k > 1$ auf verteilten Kanten). Allerdings stellte es sich als schwierig heraus, die Kanten zu berechnen, auf die das Potential verteilt werden sollte. Es war im Rahmen dieser Bachelorarbeit nicht möglich, zu dieser Problemstellung einen Algorithmus anzugeben, so dass diese Variante des Potentialproblems in Form einer zukünftigen Arbeit weiter untersucht werden kann, um evtl. einen effizienten Algorithmus zu finden.
