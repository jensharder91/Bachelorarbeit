\section{Implementierung}

Im Rahmen einer Projektarbeit im Wintersemester 2014/15 habe ich den in \cite{cima_paper} beschriebenen Algorithmus in einem Java-Applet implementiert. Ziel war es, den Algorithmus visuell darzustellen und ihn anhand des Applets in einem kurzen Vortrag vorzustellen.
\\
\\
Als Bestandteil dieser Bachelorarbeit habe ich das Applet mit der in Kapitel \ref{modifizierterAlgoChapter} beschriebenen Modifikation erweitert und die beiden Algorithmen in Kapitel \ref{kap_pot} entwickelt und ebenfalls in das Applet implementiert.


\subsection{Benutzung des Applets}

Im folgenden werde ich kurz beschreiben, wie das Applet benutzt wird:

\subsubsection*{Den Baum erstellen und ändern}

Man kann einen Baum (bzw. einen Baumknoten) einfach erstellen, indem man mit einem Linksklick in das leere Applet klickt. Sofort sollte ein Knoten erscheinen.\\
Ab jetzt kann man für jedem Knoten mit einem Linksklick auf diesen, ein Kindknoten hinzufügen. Mit einem Rechtsklick auf einen Knoten wird dieser zusammen mit dem gesamte Teilbaum gelöscht.\\
Die Kantengewichte lassen sich ebenfalls mit einem Klick verändern. Ein Linksklick erhöht das entsprechende Kantengewicht um 1, ein Rechtsklick reduziert das Kantengewicht um eins. Allerdings können die Kantengewichte nicht kleiner als 1 werden.


\subsubsection*{Das Applet benutzen}

Nachdem man einen Baum erstellt hat, gibt es mehrere Funktionen, die man auswählen kann:\\

\begin{itemize}
	\item Man kann sich die Nachrichtenberechnung sowohl von dem ursprünglichen Algorithmus (Kapitel \ref{kap_algorithmus}), als auch von der Modifikation (Kapitel \ref{modifizierterAlgoChapter}) animieren lassen, oder sofort das Ergebnis anzeigen. Bei der Animation wird immer angezeigt, wie jede einzelne Nachricht zustande kommt. Dies wird in Kurzform als Text und durch Färbung der einzelnen Elemente des Baumes getan. Nachdem alle Nachrichten verschickt wurden, berechnet jeder Knoten, wie viele Agenten benötigt werden, um den gesamten Baum von diesen Knoten aus zu dekontaminieren (dieser Wert wird am Ende in jedem Knoten angezeigt). Nach Abschluss des Algorithmus wird einer der Knoten mit minimaler Agentenzahl ausgewählt und als Homebase definiert.
	
	\item Alternativ zur Nachrichtenberechnung kann man sich die beiden Algorithmen aus Kapitel \ref{kap_pot} visuell darstellen lassen. Dazu werden ebenfalls die Nachrichten einzeln animiert (oder das Ergebnis sofort angezeigt), allerdings zeigen die Informationen (Text und Einfärbung), wie in jedem einzelnen Schritt die erweiterte Nachricht berechnet wird, die in dem jeweiligen Algorithmus benutzt wird. Man muss beachten, dass die Potentialberechnung nur mit der modifizierten Variante des Algorithmus geht. Am Ende dieses Algorithmus wird angezeigt, viele Agenten jeder Knoten zu Dekontamination benötigt, werlcher Knoten die Homebase ist, und, falls möglich, welche Kante reduziert werden sollte, um die Agentenzahl möglichst weit zu reduzieren.
	
	\item Außerdem lässt sich im Applet animieren, wie nach erfolgreicher Nachrichtenberechnung die Dekontamination des Baumes funktioniert. Die Strategie der Agenten ist dabei Monoton. Dies bedeutet, dass jeder Knoten, der schon einmal dekontaminiert wurde, nicht mehr kontaminiert werden kann, da an den entsprechenden Stellen im Baum Wachen aufgestellt werden. Sind alle Knoten dekontaminiert ist der Algorithmus abgeschlossen.
\end{itemize}

Alle Animationen können entweder als Ganzes abgespielt werden, oder in einer Step-by-step-Animation in Teilen durchgeführt werden. \\
Das Applet hat eine sehr intuitive Benutzeroberfläche. Alle Optionen, die verfügbar sind, sind sichtbar. Alle in dem Moment nicht verfügbaren Optionen werden ausgeblendet, um eine bessere Übersicht zu gewährleisten.