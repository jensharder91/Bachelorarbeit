\section{Der Algorithmus}\label{kap_algorithmus}
Im folgen werde ich zunächst den Algorithmus aus dem Paper \cite{cima_paper} beschreiben und erklären, sowie im Anschluss ein paar Modifikationen erläutern, mit denen ich in dieser Bachelorarbeit weiterarbeiten werde.\\
Der Algorithmus berechnet für einen gegebenen Baum mit Kantengewichten die Mindestanzahl an Agenten, die den Baum dekontaminieren können. Außerdem gibt er an, an welchem Knoten (Homebase) diese Agenten eingesetzt werden müssen.



\subsection{Originaler Algorithmus aus dem Artikel \cite{cima_paper}}\label{paperAlgoChapter}

Zu Beginn des Algorithmus, werden von allen Knoten ihre Knotengewichte berechnet. Dieses ergibt sich jeweils aus dem maximalen Kantengewicht aller zu diesem Knoten führenden Kanten. 

\begin{mydef}
	Für das Knotengewicht $\omega(x)$ vom Knoten x gilt: $\omega(x) = max_{e} \omega(e)$ für jede Kante e inzident zu x.
\end{mydef}

Das Knotengewicht gibt an, wie viele Agenten als Wachen benötigt werden, um den aktuellen Knoten (bzw. den dazugehörigen Teilbaum) vor einem kontaminierten Teilbaum zu schützen.
\\
\\
Nachdem für alle Knoten die entsprechenden Gewichte berechnet wurden, werden zwischen den Baumknoten Nachrichten verschickt. Diese Nachrichten beinhalten die Information, wie viele Agenten mindestens benötigt werden, um den entsprechenden Teilbaum zu dekontaminieren.  

\begin{mydef}\label{def_nachricht}
	Eine Nachricht wird von einem Knoten x zu einem Knoten y im Baum verschickt. Diese erhält Informationen über den Teilbaum von x, die von Knoten y zur Berechnung der Agenten benötigt werden.
\end{mydef}

Die Nachrichten werden nach folgender Rekursionsformel \cite{cima_paper} berechnet:\\
Sei $e = \{x, y\}$ eine Kante inzident zu $x$ und $\lambda_{x}(e)$ die Nachricht, die über $e$ nach $x$ geschickt wird. Ist $y$ ein Blatt des Baumes, so gilt $\lambda_{x}(e) = \omega(y)$. Ansonsten seien $z_{1}, ... z_{deg(y)-1}$ die Nachbarn von y außer x. Außerdem gilt $l_{i} = \lambda_{y}(\{y, z_{i}\})$ und sei o.B.d.A. $l_{i} \geq l_{i+1}$: $$\lambda_{x}(e) = max \{l_{1}, l_{2} + \omega(y)\}.$$



Da die Rekursionsgleichung etwas kryptisch wirkt, und es sowohl für die Implementierung, als auch für das allgemeine Verständnis anschaulicher ist, den Algorithmus durch eine Fallunterscheidung zu betrachten, werde ich im Folgenden drei Fälle erläutern, die im Algorithmus auftauchen können:\\
Es gibt drei verschiedene Fälle, wie Nachrichten berechnet und verschickt werden können:

\begin{enumerate}
	
	\item Fall:
		
		%		\begin{minipage}{0.55\textwidth} 
		Zu beginn des Algorithmus, oder falls keine weitere Nachricht mehr berechnet werden kann, sendet ein beliebiger Blattknoten, der noch keine Nachricht versendet hat, seine Nachricht an den eigenen Nachbarn.\\
		
		Die zu sendende Nachricht $\lambda_{y}$ vom Blatt x an seinen Nachbarknoten y ist dabei nur das eigene Gewicht (Abbildung \ref{abb_leaf}): Es gilt daher $\lambda_{y} = \omega(x)$
		%		\end{minipage}
		%		\hfill
		%		\begin{minipage}{0.35\textwidth}
		
		\includegraphics[width=\textwidth]{bilder/abb_blattknoten.png}
		\captionof{figure}{Das Knotengewicht des Blattknotens x (grün) bestimmt den Wert Nachricht $\lambda_{y} = 3$ (blau) zum Nachbarknoten y.}
		\label{abb_leaf}
		%		\end{minipage}
		
	
			
	\item Fall:
	
		Der aktuelle Knoten x hat genau n-1 Nachrichten erhalten, wobei n die Anzahl der inzidenten Knoten ist. (Also er hat von jedem Nachbarn außer einem eine Nachricht erhalten.)
		\\
	
	%				\begin{minipage}{0.50\textwidth} 
		
		Um eine Nachricht von x an den Nachbarknoten y zu senden, der bis zu diesem Zeitpunkt noch keine Nachricht an x gesendet hat, nimmt man die zwei größten Nachrichten ($l_{1}$ und $l_{2}$) die bei x angekommen sind. Mit diesen beiden angekommenen Nachrichten $l_{1} \ge l_{2}$ sowie dem Knotengewicht $\omega(x)$ wird die Nachricht $\lambda_{y}$ an y wie folgt berechnet (siehe auch Abbildung \ref{abb_n-1}): $\lambda_{y} = max\{l_{1},  l_{2} + \omega(x)\}$
		\\
		\\	
		Nach dem Berechnen und Versenden der Nachricht $\lambda_{y}$ muss x auf die letzte ankommende Nachricht (von y) warten. Sobald diese Nachricht angekommen ist, berechnet x die Nachrichten für alle anderen Nachbarknoten, wie im Fall \ref{labelAufUnterfall} beschrieben. 
	%				\end{minipage}
	%				\hfill
	%				\begin{minipage}{0.35\textwidth}
		
		\includegraphics[width=\textwidth]{bilder/abb_paper_n-1knoten.png}
		\captionof{figure}{Der Wert der neuen Nachricht $\lambda_{y}$(blau) von Knoten x zu Knoten y beträgt 6, da $l_{2} + \omega(x) = 6$ (grün) größer ist als $l_{1} = 3$ (gelb).}
		\label{abb_n-1}
	%				\end{minipage}
		
		
	\item Fall:
	
		Der aktuelle Knoten x hat bereits von allen n zu ihm inzidenten Knoten eine Nachricht erhalten, selber jedoch noch nicht zu allen Knoten eine Nachricht gesendet. Diese Nachrichten müssen nun berechnet und versendet werden:\label{labelAufUnterfall}
		\\
		
%				\begin{minipage}{0.50\textwidth} 
		
		Um eine Nachricht von x an einen Nachbarknoten y zu senden, der bis zu diesem Zeitpunkt noch keine Nachricht von x erhalten hat, nimmt man die 2 größten Nachrichten ($l_{1}$ und $l_{2}$), die bereits bei x angekommen sind. Dabei ist es wichtig, dass weder $l_{1}$ noch $l_{2}$ von y stammen. Die Nachricht von y wird bei dieser Nachrichtenberechnung ignoriert! Mit den beiden ermittelten Nachrichten $l_{1} \ge l_{2}$ sowie dem Knotengewicht $\omega(x)$ wird die Nachricht $\lambda_{y}$ an y wie folgt berechnet (Abbildung \ref{abb_n}):  $\lambda_{y} = max\{l_{1},  l_{2} + \omega(x)\}$
		\\
		\\
		Diese Berechnung wird für jeden Nachbarknoten von x wiederholt, so dass jeder Nachbar eine Nachricht von x erhält. Nach diesem Schritt ist der Knoten x mit dem Algorithmus fertig, da er sowohl von jedem Nachbar eine Nachricht erhalten hat, als auch an jeden Nachbarn eine Nachricht gesendet hat.
%				\end{minipage}
%				\hfill
%				\begin{minipage}{0.35\textwidth}
		
		\includegraphics[width=\textwidth]{bilder/abb_paper_nknoten.png}
		\captionof{figure}{Die neue Nachricht $\lambda_{y}$ (blau) von Knoten x zu Knoten y hat den Wert 7, da $l_{1} = 7$ (grün) größer ist als $l_{2} + \omega(x) = 6$ (gelb).}
		\label{abb_n}
%				\end{minipage}
	
		
		
\end{enumerate}


%\subsubsection*{Berechnung der minimalen Agenten}

Sobald alle Knoten sowohl an alle Nachbarn eine Nachricht geschickt haben, als auch von allen Nachbarn eine Nachricht erhalten haben (jede Kante im gesamten Baum überträgt genau zwei Nachrichten), hat jeder Knoten alle Informationen, die gebraucht werden, um die minimale Anzahl an Agenten zu errechnen, die für diesen Knoten als Homebase benötigt werden.

\begin{mydef}
	Die Anzahl an Agenten $\mu(x)$ ist die Agentenzahl, welche benötigt wird, um den gesamten Baum vom Knoten x aus zu dekontaminieren. $\mu = \min \mu(i)$ für alle Knoten i im Baum, ist die minimale Anzahl an Agenten, die diesen Baum von der Homebase ausgehend dekontaminieren können.
\end{mydef}

\begin{mydef}\label{def_homebase}
	Die Homebase ein beliebiger Knoten x im Baum, für den gilt: $\mu(x) = \mu$.
\end{mydef}

Die minimale Agentenanzahl $\mu(x)$, die am Knoten x benötigt werden, wird wie folgt berechnet:
\\
$\mu(x) = max\{l_{1},  l_{2} + \omega(x)\}$, wobei analog zu der Berechnung der Nachrichten gilt: $l_{1} \ge l_{2}$ sind die größten beiden angekommenen Nachrichten und $\omega(x)$ ist das Knotengewicht von x.
\\
\\
Nachdem alle Knoten i ihr $\mu(i)$ berechnet haben, wird ein Knoten mit minimalen $\mu$ ausgewählt. Dieser ist die neue Homebase, von dem aus $\mu$ viele Agenten den gesamten Baum dekontaminieren können, ohne die Monotonie-Eigenschaft zu verletzen (siehe Definition \ref{def_monotonie}).

\newpage

\subsection{Modifikationen am Algorithmus}\label{modifizierterAlgoChapter}


	\begin{wrapfigure}{l}{0.7\textwidth}
		\begin{center}
			\includegraphics[width=0.7\textwidth]{bilder/abb_paper_problem.png}
		\end{center}
		\caption{Problem mit dem Algorithmus aus dem Paper. Der Knoten x sollte intuitiv mit 5 Agenten auskommen}
		\label{fig:negBeispielPaperAlgo}
	\end{wrapfigure}

Die Ergebnisse des im Artikel beschriebenen Algorithmus stimmt in einigen Fällen nicht mit der Intuition überein, die man bekommt, wenn man einige Beispielbäume betrachtet (z.B. Abbildung \ref{fig:negBeispielPaperAlgo}). In dem  angegebenen Beispiel sollte der Knoten x mit fünf Agenten auskommen: Alle fünf Agenten gehen über die erste Kante zum mittleren Knoten. Von dort aus werden nur noch zwei Agenten benötigt: Einer hält Wache, um die Monotonie zu gewährleisten, und der zweite dekontaminiert ein Blatt. Zum Schluss wird das letzte Blatt von einem Agenten dekontaminiert und der Algorithmus ist fertig.
\\
Der Unterschied zwischen dem originalen Algorithmus und der Intuition ist folgender: Der originale Algorithmus plant im Beispiel für den mittleren Knoten fünf Wachen ein (das Knotengewicht ist 5), obwohl wir über die Kante mit Gewicht 5 gekommen sind, diese dadurch dekontaminiert ist, und wir sie nicht mehr bewachen müssen. Es werden dadurch mehr Agenten eingeplant, als wirklich benötigt werden.
\\
Da das Problem, wie erläutert, am Knotengewicht liegt, werde ich im folgenden eine Variante des Algorithmus beschreiben, die etwas anders mit dem Knotengewicht umgeht. Allerdings kommt es zu Fehlern, wenn man nur das Knotengewicht verändert, weshalb man zusätzliche Fälle in den Algorithmus einbauen muss, damit dieser sowohl richtig, als auch intuitiv sinnvoll funktioniert:
\\
\\
Im Gegensatz zum originalen Algorithmus wird im modifizierten Algorithmus nicht mehr das Knotengewicht in die direkte Berechnung der Nachricht miteinbezogen, sondern dient nur noch dazu, zu überprüfen, dass die berechnete Nachricht nicht zu klein ist.
\\
Dazu muss der Algorithmus in Kapitel \ref{paperAlgoChapter} wie folgt abgeändert werden:

\subsubsection*{Modifizierte Berechnung der Nachricht:}

\begin{enumerate}[label=\alph*)]
	
	\item 
		Um eine Nachricht an einen Nachbarknoten y zu senden, der bis zu diesem Zeitpunkt noch keine Nachricht von x erhalten hat, nimmt man das Kantengewicht der zwei größten Kanten ($edge_{1}$ und $edge_{2}$) die inzident zu x sind, jedoch nicht zu y. Mit den beiden ermittelten Kantengewichten, für die gilt $edge_{1} \ge edge_{2}$ wird die Nachricht $\lambda_{y}$ an y wie auch in Abbildung \ref{modifiziert_a} gezeigt berechnet:\\
		$\lambda_{y} = edge_{1} + edge_{2}$.
		\\
		\\
		Vor der Bestimmung der zwei maximalen Kanten werden $edge_{1}$ und $edge_{2}$ auf 0 initialisiert, sodass die Berechnung auch klappt, falls es weniger als zwei Kanten gibt, auf die die Bedingung inzident zu x und nicht inzident zu y  zutrifft.
	
	\item
		Nachdem die neue Nachricht $\lambda_{y}$ nun berechnet wurde, müssen wir noch überprüfen, dass $\lambda_{y}$ nicht kleiner ist als das Knotengewicht $\omega(x)$. Ist dies der Fall bedeutet dass, dass das Kantengewicht der Kante, über die wir die Nachricht nach y schicken größer ist als die berechnete Nachricht. Da dies nicht sein darf, da wir sonst nicht genug Agenten haben würden, um über die Kante zu laufen, müssen wir die Nachricht die wir an y schicken entsprechend anpassen (Beispiel in Abbildung \ref{modifiziert_b}):
		
		\begin{algorithmic}
			\If {$\lambda_{y} \leq \omega(x)$}
			\State $\lambda_{y} \gets \omega(x)$
			\EndIf
		\end{algorithmic}
	
	\item
		Außerdem darf die berechnete Nachricht $\lambda_{y}$ nicht kleiner sein als die größte in x angekommene Nachricht $l_{1}$ (außer der Nachricht, die wir evt. schon von y erhalten haben). Ist die Nachricht $\lambda_{y}$ kleiner als $l_{1}$, würden wir nicht genug Agenten bekommen, um den Teilbaum, aus dem $l_{1}$ kommt zu dekontaminieren. Wir müssen $\lambda_{y}$ kontrollieren und evt. wie in Abbildung \ref{modifiziert_c} anpassen:
		
		\begin{algorithmic}
			\If {$\lambda_{y} \leq l_{1}$}
			\State $\lambda_{y} \gets l_{1}$
			\EndIf
		\end{algorithmic}
	
\end{enumerate}

%\begin{figure}
%	\floatbox[{\capbeside\thisfloatsetup{capbesideposition={right,top},capbesidewidth=4cm}}]{figure}[\FBwidth]
%	{\caption{A test figure with its caption side by side}\label{fig:test}}
%	{\includegraphics[width=0.75\textwidth]{bilder/abb_neu_max1max2.png}}
%\end{figure}
%
%\begin{figure}
%	\centering
%	\begin{subfigure}[b]{0.3\textwidth}
%		\floatbox[{\capbeside\thisfloatsetup{capbesideposition={right,top},capbesidewidth=4cm}}]{figure}[\FBwidth]
%		{\caption{A test figure with its caption side by side}\label{fig:test}}
%		{\includegraphics[width=0.75\textwidth]{bilder/abb_neu_max1max2.png}}
%	\end{subfigure}
%	
%	\begin{subfigure}[b]{0.3\textwidth}
%		\floatbox[{\capbeside\thisfloatsetup{capbesideposition={right,top},capbesidewidth=4cm}}]{figure}[\FBwidth]
%		{\caption{A test figure with its caption side by side}\label{fig:test}}
%		{\includegraphics[width=0.75\textwidth]{bilder/abb_neu_max1max2.png}}
%	\end{subfigure}
%	
%	\begin{subfigure}[b]{0.3\textwidth}
%		\floatbox[{\capbeside\thisfloatsetup{capbesideposition={right,top},capbesidewidth=4cm}}]{figure}[\FBwidth]
%		{\caption{A test figure with its caption side by side}\label{fig:test}}
%		{\includegraphics[width=0.75\textwidth]{bilder/abb_neu_max1max2.png}}
%	\end{subfigure}
%	\caption{Pictures of animals}\label{fig:animals}
%\end{figure}


\begin{figure}[h]
	\subfigure[Die Nachricht $\lambda_{y}$ (blau) von x zu y hat den Wert 5, da $edge_{1} $ und $edge_{2}$ (beide grün) die entscheidenden Kanten sind.
	\label{modifiziert_a}]{\includegraphics[width=0.65\textwidth]{bilder/abb_neu_max1max2.png}}
%	\hfill  
%lösche die neue line und füge das \hfill wieder ein um die bilder nebeneiander zu haben
	
	\subfigure[Da das Knotengewicht durch die Kante zwischen x und y bestimmt wird (grün) und dieses größer ist als die Summe der beiden normalen Kanten $edge_{1}$ und $edge_{2}$ (gelb), bestimmt diese Kante den Wert der Nachricht $\lambda_{y} = 5$ (blau). \label{modifiziert_b}]{\includegraphics[width=0.65\textwidth]{bilder/abb_neu_edge.png}} 
%	\hfill
	
	\subfigure[Die größte Nachricht (grün), die an x ankommt ist größer als die beiden Kanten $edge_{1}$ und $edge_{2}$ (gelb) und bestimmt dadurch den Nachrichtenwert $\lambda_{y} = 8$ (blau). \label{modifiziert_c}]{\includegraphics[width=0.65\textwidth]{bilder/abb_neu_msgData.png}}  
	
	\caption{Drei Fälle, die bei der Modifizierung des Algorithmus beachtet werden müssen, um alle Spezialfälle abzudecken.} 

\end{figure}
	
	\begin{theorem}\label{thm_laufzeit_modifikation}
		Die Modifikationen des Algorithmus ändern nichts an der linearen Laufzeit.
	\end{theorem}
	\begin{proof}
		Die Grundidee des Algorithmus bleibt erhalten. Jeder Knoten schickt zu all seinen Nachbarn je eine Nachricht. Die Anzahl der Nachrichten ändert sich durch die Modifikation nicht, sondern nur die Berechnung an sich.\\Die Berechnung kann in konstanter Zeit durchgeführt werden, dass pro Knoten nur drei Schritte notwendig sind:
		\begin{enumerate}
			\item Berechnung der Nachricht durch einfache Addition
			\item Vergleich der Nachricht (und evtl. Anpassen) mit dem Knotengewicht
			\item Vergleich der Nachricht (und evtl. Anpassen) mit der maximalen Nachricht.
		\end{enumerate}
		Die lineare Zeit des als Grundlage verwendeten Algorithmus ist im erwähnten Paper bereits bewiesen worden.
	\end{proof}



