\documentclass{beamer}
\usepackage{amsmath}
\usepackage{amsfonts}
\usepackage{amssymb}
\usepackage[utf8]{inputenc}
\usepackage[german]{babel}
%\usetheme{Madrid}Goettingen  Marburg   
%\usetheme{Amsterdam}
\usetheme{Madrid}


%für zitat
\usepackage{microtype}
\usepackage{mdframed}

%caption next to figure
\usepackage{floatrow}
\usepackage{graphicx}

%text in ecke
\usepackage{eso-pic}

%figure in frame
\usepackage{capt-of}


\title{Gewässerschutz in Deutschland}
\author{Jens Harder}
\date{21. Mai 2015}

%definition zitat
\mdfdefinestyle{zitat}{
	hidealllines=true,leftline=true,linewidth=1.5pt,
	leftmargin=2.0em,innerleftmargin=1.0em,rightmargin=3.0em,
	innerrightmargin=0pt,innerbottommargin=0.5em, innertopmargin=0.5em}


\begin{document}
	
	\begin{frame}
		\titlepage
	\end{frame}
	
	\begin{frame}
		\tableofcontents
	\end{frame}
	
	\section{Einleitung}
	{% Ab hier wird das Bild Hintergrundfüllend eingefügt
		\setbeamertemplate{background canvas}{
			\includegraphics[%
			width=\paperwidth,
			height=\paperheight]{bilder/wasserschutzgebiet.jpg}
		}
		\begin{frame}
			\frametitle{Einleitung}
			\large
			
			\begin{mdframed}[style=zitat]
				\textbf{Definition Gewässerschutz:}\\
				Unter Gewässerschutz versteht man alle Maßnahmen, welche oberirdische Gewässer, Küstengewässer, sowie das Grundwasser vor Verunreinigung schützen.
			\end{mdframed}
			
			\AddToShipoutPictureFG*{
				\AtPageLowerLeft{\put(0,12){\makebox[\paperwidth][l]{\cite{bildWasserschutzgebiet}}}}
		   }%
			
		\end{frame}
	}% hier ended die Gültigkeit des Bildeinfügens
	
	\begin{frame}
		\frametitle{Einflüsse auf die Wasserqualität}
		\LARGE
		\begin{itemize}
			\item Beschaffenheit des (Gewässer-)Einzugsgebietes\\ {\normalsize (Größe, Oberflächenbeschaffenheit, Niederschlag, Untergrund, Flächennutzung, Besiedlung)}
			
			\item Sauerstoffgehalt des Gewässers		
				
			\item Selbstreinigungskraft des Gewässers
			
			\item zugefügte Verunreinigung

		\end{itemize}
	\end{frame}
		
	\section{Warum brauchen wir Gewässerschutz?}
	\begin{frame}
		\frametitle{Warum brauchen wir Gewässerschutz?}
		\LARGE
		
		\textbf{"'Wasser ist keine übliche Handelsware, sondern ein ererbtes Gut, das geschützt, verteidigt und entsprechend behandelt werden muss."'} \cite{wrrl}
		
%		Ohne Wasser kein Leben 
		
	\end{frame}
	
	\begin{frame}
		\frametitle{Warum brauchen wir Gewässerschutz? \\ - Wasser als Lebensgrundlage - }
		\LARGE
		\begin{itemize}
			\item Trinkwasserversorgung 
			\item Bewässerung für die Landwirtschaft
			\item Wasserversorgung für die Industrie
			\item Schifffahrt
			\item Erzeugung von Strom
			\item Tourismus
			\item Gewinnung von Kies als Baumaterial
		\end{itemize}
	\end{frame}
	
	\begin{frame}
		\frametitle{Warum brauchen wir Gewässerschutz? \\ - Gewässerschutz ist notwendig! - }
		\LARGE
		\begin{itemize}
			\item Ohne Gewässerschutz wären die Gewässer der Wirtschaft ausgeliefert (Kompromiss)
			\item Ein "'gesundes"' Gewässer ist robuster gegenüber Schäden
			\item Sauberes Trinkwasser ist überlebensnotwendig
		\end{itemize}
		
	\end{frame}
	
	\section{Gesetzliche Richtlinien}
	\begin{frame}
		\frametitle{Gesetzliche Richtlinien}
		\LARGE
		
		\begin{itemize}
			\item EG-Wasserrahmenrichtlinie (WRRL) bzw. Richtlinie 2000/60/EG
			
			\begin{itemize}
				\LARGE
				\item Ziel: "'Guter Zustand"' der Gewässer
				
				\item EG-Nitratrichtlinie ("'Düngeverordnung"')
			\end{itemize}
			
			\item Wasserhaushaltsgesetz (WHG)
		\end{itemize}
	\end{frame}
	
	\section{Die Landwirtschaft als Mitverursacher für die Verschmutzung}
	\begin{frame}
		\frametitle{Die Landwirtschaft als Mitverursacher für die Verschmutzung}
		\LARGE
		
		\begin{itemize}
			\item Landwirtschaft als sog. offenes System
			
			\item Großteil des Düngers wird nicht von Pflanzen aufgenommen ("'Überdüngung"')
			
			\item Gülle von Viehbeständen in Gewässern schädlich
			
			\item häusliche oder industrielle Abwasser
		\end{itemize}
		
	\end{frame}
	
	\begin{frame}
		\frametitle{Stickstoff als Hauptproblem für die Verschmutzung}
		\LARGE
		
		\begin{itemize}
			\item Stickstoff ist lebensnotwendig
			
			\item $N_{2}$ hat 78\% Luftanteil
			
			\item $N_{2}$  reaktionsträge und damit absolut unschädlich
			
			\item Haber-Bosch-Verfahren: $N_{2} + 3H_{2} \rightarrow 2NH_{3}$ (aus Stickstoff wird Mineraldünger)
		\end{itemize}
				
	\end{frame}
	
	\begin{frame}
		\frametitle{Eutrophierung}
		\large
		
		\begin{mdframed}[style=zitat]
			\textbf{Definition Eutrophierung:} \\ 
			Unter Eutrophierung versteht man die unerwünschte Anreicherung eines Gewässers an Nährstoffen, besonders Phosphor- und Stickstoffverbindungen. Dies führt zu einer Übervermehrung von Pflanzen (v.a. Algen) und tierischem Plankton durch "'Überdüngung"'.% \cite[Band 3, S.544]{brockhaus}.
		\end{mdframed}
		
		
	\end{frame}
	
	{% Ab hier wird das Bild Hintergrundfüllend eingefügt
		\setbeamertemplate{background canvas}{
			\includegraphics[%
			width=\paperwidth,
			height=\paperheight]{bilder/eutrophierung2.jpg}
		}
		\begin{frame}[plain]
			\frametitle{Eutrohphierung}
			\LARGE
			
			\begin{itemize}
				\item \textcolor{white}{starkes Algen- und Pflanzenwachstum}
				\item \textcolor{white}{Verminderung des Sauerstoffgehaltes}
				\item \textcolor{white}{Abschirmung des Sonnenlichtes}
				\item \textcolor{white}{Gewässer "'kippt"' durch anaerobe Abbauprozesse (Methan, Schwefelwasserstoff)}
				\item \textcolor{white}{sauerstofffreie und unbelebte Tiefenzonen}
			\end{itemize}
			
		\AddToShipoutPictureFG*{
			\AtPageLowerLeft{\put(0,3){\makebox[\paperwidth][l]{\textcolor{white}{\cite{eutrophierterFluss}}}}}
		}%
		
		\end{frame}
		
	}% hier ended die Gültigkeit des Bildeinfügens
	
	\begin{frame}
		\frametitle{Weitere Verschmutzungsursachen}
		\LARGE
		\begin{itemize}
			\item Sink- und Schwimmstoffe
			\item Krankheitserreger aus Fäkalien
			\item zersetzungsunfähige Stoffe
			\item chemisch/phyikalisch bedenkliche Stoffe \\{\normalsize (Gifte, radioaktive Stoffe, ...)} 
		\end{itemize}
	\end{frame}

	
	\begin{frame}
		\frametitle{Der langsame Weg der Verschmutzung}
		\LARGE
		
		\begin{itemize}
			\item Belastung des Grundwassers
			\begin{itemize}
				\LARGE
				\item 70\% unseres Trinkwassers
				
				\item langsame Fließgeschwindigkeit
				
				\item Wechselwirkungen mit anderen Oberflächengewässern
				
				\item diffuse Stoffeinträge
			\end{itemize}
			
		\end{itemize}
	\end{frame}
	
	\begin{frame}
		\frametitle{Der langsame Weg der Verschmutzung}
		
%		\includegraphics[width=\textwidth]{bilder/nitratbelastung_grundwasser.PNG}\\
%		test
		
		\centering
		\includegraphics[width=0.95\textwidth]{bilder/nitratbelastung_grundwasser.PNG}
		\captionof{figure}{Nitratbelastung im Grundwasser unter Wald, Grünland, Siedlungen und Acker \cite[S.13]{gewLandwirtschaft}.}
		\label{abb_nitratbelastung}
		 
	\end{frame}
	
	\begin{frame}
		\frametitle{Der langsame Weg der Verschmutzung}
		\LARGE
		
		\begin{itemize}
			\item Belastung von Flüssen und Seen
			\begin{itemize}
				\LARGE
				\item Wechselwirkungen mit dem Grundwasser
				
				\item weniger punktuelle Schadstoffquellen
				
				\item Nur 16\% hatten 2008 Güteklasse I oder II
				
			\end{itemize}
			
		\end{itemize}
		
	\end{frame}
	
	\begin{frame}
		\frametitle{Der langsame Weg der Verschmutzung}
		
		\centering
		\includegraphics[width=0.77\textwidth]{bilder/punkt_diffuse_quellen2.PNG}
		\captionof{figure}{Stickstoffeinträge aus punktuellen und diffusen Quellen in Oberflächengewässern \cite[S.14]{gewLandwirtschaft}.}
		\label{abb_punkt_diffuse_quellen}
		
	\end{frame}
	
	\begin{frame}
		\frametitle{Der langsame Weg der Verschmutzung}
		\LARGE
		
		\begin{itemize}
			\item Belastung der Küstengewässer und Meere
			\begin{itemize}
				\LARGE
				\item schädliche Stoffe aus Flüssen
				
				\item weltweiter Fischbedarf
				
				\item Tourismus
				
				\item z.B. Algenpest
				
			\end{itemize}
			
		\end{itemize}
		
	\end{frame}
	
	\section{Gewässerschonende Maßnahmen}
	\begin{frame}
		\frametitle{Gewässerschonende Maßnahmen}
		\LARGE
		
		\begin{itemize}
			\item Größte Aufgabe: Grundwasser schützen
			
			\item Weiterbildung der Landwirte
			
			\item Mischung aus:
			
			\begin{itemize}
				\LARGE
				\item Machtstrategie \normalsize (Gesetze und Richtlinien)
				
				\LARGE
				\item Förderstrategie \normalsize (finanzielle Förderungen/Anreize)
				
				\LARGE
				\item Kommunikative Strategie \normalsize(Empfehlungen, Beratungen)
			\end{itemize}
				
		\end{itemize}
		
	\end{frame}
	
	\begin{frame}
		\frametitle{Düngen}
		\LARGE
		
		\begin{itemize}
			\item "'bedarfsorientiert, effizient
			und verlustarm"' düngen
			
			\item Nährstoffbedarfsermittlung
			
			\item Hoftorbilanz
		\end{itemize}
	\end{frame}
	
	\begin{frame}
		\frametitle{Verbesserte Flächennutzung}
		\LARGE
		
		\begin{columns}[T] % align columns
			\begin{column}{.48\textwidth}
				\begin{itemize}
					\item Wasserschutzgebiete
					
					\item Gewässerrandstreifen
					
					\item Zwischenfrüchte / Untersaaten
				\end{itemize}
			\end{column}%
			\hfill%
			\begin{column}{.48\textwidth}
				\normalsize
				%		\centering
				\includegraphics[width=\textwidth]{bilder/uferrandstreifen_feld.PNG}
				\captionof{figure}{Gewässerrandstreifen, um die Oberflächengewässer zu schonen \cite{bildUferrandstreifen}. }
				\label{abb_uferschutzstreifen_feld}
			\end{column}%
		\end{columns}

	\end{frame}
	
	\begin{frame}
		\frametitle{Weitere Maßnahmen für saubere Flüsse}
		\LARGE
		
		\begin{itemize}
			\item z.B. phosphatfreie Wasch- und Reinigungsmittel
			
			\item Ausbau bzw. Neubau von Kläranlagen (Reduzierung der Schwermetallbelastung um 90\%)
			
			\item viele Maßnahmen sind schwer umzusetzen
		\end{itemize}
		
	\end{frame}
	
	\section{Beispiel: Unser Trinkwasser in Bonn}
	\begin{frame}
		\frametitle{Beispiel: Unser Trinkwasser in Bonn}
		\LARGE
		Trinkwasserversorgung in Bonn:\\
		\begin{itemize}
			\item Wahnbachtalsperre
			\item Grundwasserwerk Hennefer Siegbogen
			\item Grundwasserwerk Meindorf
		\end{itemize}
		
	\end{frame}
	
	
	{% Ab hier wird das Bild Hintergrundfüllend eingefügt
		\setbeamertemplate{background canvas}{
			\includegraphics[%
			width=\paperwidth,
			height=\paperheight]{bilder/wahnbachtalsperre2_ab.jpg}
		}
		\begin{frame}[plain]
			\frametitle{Wahnbachtalsperre}
			
			\AddToShipoutPictureFG*{
				\AtPageUpperLeft{\put(247,-40){\makebox[\paperwidth][l]{\cite{bildWahnbachtalsperre}}}}
			}%

		\end{frame}
	}% hier ended die Gültigkeit des Bildeinfügens
	
	\begin{frame}
		\frametitle{Wahnbachtalsperre}
		\LARGE
		\begin{itemize}
			
			\item 3 Gewässerschutzzonen
				\begin{itemize}
					\large
					\item Schutzzone I: Uferbereich
					\item Schutzzone II: engere Zone
					\item Schutzzone III: weitere Zone
				\end{itemize}
			
			\item seit 1977/78 Phosphor-Eliminierungsanlage (PEA)
			
			\item Kooperation mit der Landwirtschaft
		\end{itemize}
	\end{frame}
	
	{% Ab hier wird das Bild Hintergrundfüllend eingefügt
		\setbeamertemplate{background canvas}{
			\includegraphics[%
			width=\paperwidth,
			height=\paperheight]{bilder/grundwasserwerk_unter_sieg.JPG}
		}
	\begin{frame}
		
		\AddToShipoutPictureFG*{
			\AtPageUpperLeft{\put(267,-10){\makebox[\paperwidth][l]{\textcolor{white}{\cite{wahnbachtalsperre}}}}}
		}%

	\end{frame}
	}% hier ended die Gültigkeit des Bildeinfügens
	
	\begin{frame}
		\frametitle{Wasseraufbereitung}
		\LARGE
		Wasserziel: klar, wohlschmeckend, frei von Krankheitserregern und Schadstoffen:
		\begin{enumerate}
			\item Flockung
			\item Mehrschichtfilter
			\item Ultraschall
			\item UV-Licht
			\item Kalkwasser gemischt
		\end{enumerate}
	\end{frame}
	
	
	\section{Ausblick auf die Zukunft und Fazit}
	\begin{frame}
		\frametitle{Ausblick auf die Zukunft und Fazit}
		\LARGE
		
		\begin{itemize}
			\item Klimawandel
			
			\begin{itemize}
				\LARGE
				\item äußere Gegebenheiten ändern sich
				
				\item wahrscheinlich mehr Wetterextreme
				
				\item Landwirtschaft muss sich anpassen
				
				\item wasserschonender Umgang mit Resscourcen
			\end{itemize}
		\end{itemize}
	\end{frame}
	
	\begin{frame}
		\frametitle{Ausblick auf die Zukunft und Fazit}
		\LARGE
		
		\begin{itemize}
			\item Wasseraufbereitung
			
			\begin{itemize}
				\LARGE
				\item sehr aufwendige Wasseraufbereitung
				
				\item Trinkwasserversorgung als Standortvorteil
				
				\item jeder einzelne ist beim Gewässerschutz mitverantwortlich
			\end{itemize}
		\end{itemize}
	\end{frame}
	
	\begin{frame}
		\frametitle{Ausblick auf die Zukunft und Fazit}
		\LARGE
		
		\textbf{"'Wasser ist keine Ware, sondern ein wesentlicher Bestandteil unseres Lebens, auf den wir nicht verzichten können"'} \cite{wahnbachtalsperre}

		
	\end{frame}
	
	\begin{frame}
		\begin{center}
			\LARGE \textbf{Vielen Dank für Ihre Aufmerksamkeit!}
		\end{center}
	\end{frame}

	
	% Literaturliste endgueltig anzeigen
	\begin{thebibliography}{99}
		
		
		\bibitem[\textsc{Boland} et al. 2003]{landBildung} \textsc{H. Boland, V. Hoffmann, U. J. Nagel} (2003): Landwirtschaftliche Bildung und Beratung zum Gewässerschutz in Deutschland. (Margraf Verlag) Weikersheim.
		
		\bibitem[Brockhaus 2005]{brockhaus} Brockhaus (2005): Brockhaus A-Z Wissen. Band 3 und 4. (F.A. Brockhaus) Leipzig/Mannheim.
		
		\bibitem[\textsc{Friedl} 2004]{wasserrahmrichtlinie} \textsc{Dipl. Ing. Ch. Friedl} (2004): Die Wasserrahmenlichtlinie - Neues Fundament für den Gewässerschutz in Europa. (Bundesministerium für Umwelt, Naturschutz und Reaktorsicherheit) Berlin.
		
		\bibitem[\textsc{Köhler} 1996]{lebensaderRhein} \textsc{Dr. E. Köhler} (1996): Lebensader Rhein. (Vereinigung Deutscher Gewässerschutz e.V.) Bonn.
		
		\bibitem[\textsc{Mohupt} et al. 2010]{gewLandwirtschaft} \textsc{V. Mohupt, J. Rechenberg, S. Richter, D. Schulz, R. Wolter} (2010): Gewässerschutz mit der Landwirtschaft. (Umwelt Bundes Amt) Berlin.
	
		
		\bibitem[Wahnbachtalsperre]{wahnbachtalsperre} Die Wahnbachtalsperre: http://www.wahnbachwasser.de/ (letzter Aufruf 17.05.2015)
		
		\bibitem[Wasserschutzzonen]{wasserschutzzonen} Die Wasserschutzzonen: http://www.kreis-tir.de/umwelt/wasserrecht/wasserschutzgebiete.html (letzter Aufruf 16.05.2015)
		
		\bibitem[Wasserrahmenrichtlinie]{wrrl} Richtlinie 2000/60/EG (Wasserrahmenrichtlinie): http://eur-lex.europa.eu/LexUriServ/LexUriServ.do?uri=CONSLEG:2000L0060:20011216:DE:PDF (letzter Aufruf 17.05.2015)
		
		\bibitem[Bild eutrophierter Fluss]{eutrophierterFluss} Bild eutrophierter Fluss: http://weltschaukasten.de/wp-content/uploads/DSC0067a\_bearbeitet-1.jpg (letzter Aufruf 17.05.2015)
		
		\bibitem[Bild Uferrandstreifen]{bildUferrandstreifen} Bild Uferrandstreifen: http://www.emslandbiber.de/bilder biologie/uferrandstreifen.jpg (letzter Aufruf 17.05.2015)
		
		\bibitem[Bild Wahnbachtalsperre]{bildWahnbachtalsperre} Bild Wahnbachtalsperre: http://www.ksta.de/image/view/2012/10/15/20883258,16552674,highRes,maxh,480,maxw,480,Naafbachsperre-01.jpg (letzter Aufruf 15.05.2015)
		
		\bibitem[Bild Wasserschutzgebiet]{bildWasserschutzgebiet} Bild Wasserschutzgebiet: http://www.urbane-landwirtschaft.org/sites/www.urbane-landwirtschaft.org/files/Wasserschutzgebiet.jpg (letzter Aufruf 19.05.2015)
		
	\end{thebibliography} 

	
\end{document}